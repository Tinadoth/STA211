% Options for packages loaded elsewhere
\PassOptionsToPackage{unicode}{hyperref}
\PassOptionsToPackage{hyphens}{url}
%
\documentclass[
]{article}
\usepackage{amsmath,amssymb}
\usepackage{lmodern}
\usepackage{ifxetex,ifluatex}
\ifnum 0\ifxetex 1\fi\ifluatex 1\fi=0 % if pdftex
  \usepackage[T1]{fontenc}
  \usepackage[utf8]{inputenc}
  \usepackage{textcomp} % provide euro and other symbols
\else % if luatex or xetex
  \usepackage{unicode-math}
  \defaultfontfeatures{Scale=MatchLowercase}
  \defaultfontfeatures[\rmfamily]{Ligatures=TeX,Scale=1}
\fi
% Use upquote if available, for straight quotes in verbatim environments
\IfFileExists{upquote.sty}{\usepackage{upquote}}{}
\IfFileExists{microtype.sty}{% use microtype if available
  \usepackage[]{microtype}
  \UseMicrotypeSet[protrusion]{basicmath} % disable protrusion for tt fonts
}{}
\makeatletter
\@ifundefined{KOMAClassName}{% if non-KOMA class
  \IfFileExists{parskip.sty}{%
    \usepackage{parskip}
  }{% else
    \setlength{\parindent}{0pt}
    \setlength{\parskip}{6pt plus 2pt minus 1pt}}
}{% if KOMA class
  \KOMAoptions{parskip=half}}
\makeatother
\usepackage{xcolor}
\IfFileExists{xurl.sty}{\usepackage{xurl}}{} % add URL line breaks if available
\IfFileExists{bookmark.sty}{\usepackage{bookmark}}{\usepackage{hyperref}}
\hypersetup{
  pdftitle={Project2},
  pdfauthor={Maribeth McCook},
  hidelinks,
  pdfcreator={LaTeX via pandoc}}
\urlstyle{same} % disable monospaced font for URLs
\usepackage[margin=1in]{geometry}
\usepackage{color}
\usepackage{fancyvrb}
\newcommand{\VerbBar}{|}
\newcommand{\VERB}{\Verb[commandchars=\\\{\}]}
\DefineVerbatimEnvironment{Highlighting}{Verbatim}{commandchars=\\\{\}}
% Add ',fontsize=\small' for more characters per line
\usepackage{framed}
\definecolor{shadecolor}{RGB}{248,248,248}
\newenvironment{Shaded}{\begin{snugshade}}{\end{snugshade}}
\newcommand{\AlertTok}[1]{\textcolor[rgb]{0.94,0.16,0.16}{#1}}
\newcommand{\AnnotationTok}[1]{\textcolor[rgb]{0.56,0.35,0.01}{\textbf{\textit{#1}}}}
\newcommand{\AttributeTok}[1]{\textcolor[rgb]{0.77,0.63,0.00}{#1}}
\newcommand{\BaseNTok}[1]{\textcolor[rgb]{0.00,0.00,0.81}{#1}}
\newcommand{\BuiltInTok}[1]{#1}
\newcommand{\CharTok}[1]{\textcolor[rgb]{0.31,0.60,0.02}{#1}}
\newcommand{\CommentTok}[1]{\textcolor[rgb]{0.56,0.35,0.01}{\textit{#1}}}
\newcommand{\CommentVarTok}[1]{\textcolor[rgb]{0.56,0.35,0.01}{\textbf{\textit{#1}}}}
\newcommand{\ConstantTok}[1]{\textcolor[rgb]{0.00,0.00,0.00}{#1}}
\newcommand{\ControlFlowTok}[1]{\textcolor[rgb]{0.13,0.29,0.53}{\textbf{#1}}}
\newcommand{\DataTypeTok}[1]{\textcolor[rgb]{0.13,0.29,0.53}{#1}}
\newcommand{\DecValTok}[1]{\textcolor[rgb]{0.00,0.00,0.81}{#1}}
\newcommand{\DocumentationTok}[1]{\textcolor[rgb]{0.56,0.35,0.01}{\textbf{\textit{#1}}}}
\newcommand{\ErrorTok}[1]{\textcolor[rgb]{0.64,0.00,0.00}{\textbf{#1}}}
\newcommand{\ExtensionTok}[1]{#1}
\newcommand{\FloatTok}[1]{\textcolor[rgb]{0.00,0.00,0.81}{#1}}
\newcommand{\FunctionTok}[1]{\textcolor[rgb]{0.00,0.00,0.00}{#1}}
\newcommand{\ImportTok}[1]{#1}
\newcommand{\InformationTok}[1]{\textcolor[rgb]{0.56,0.35,0.01}{\textbf{\textit{#1}}}}
\newcommand{\KeywordTok}[1]{\textcolor[rgb]{0.13,0.29,0.53}{\textbf{#1}}}
\newcommand{\NormalTok}[1]{#1}
\newcommand{\OperatorTok}[1]{\textcolor[rgb]{0.81,0.36,0.00}{\textbf{#1}}}
\newcommand{\OtherTok}[1]{\textcolor[rgb]{0.56,0.35,0.01}{#1}}
\newcommand{\PreprocessorTok}[1]{\textcolor[rgb]{0.56,0.35,0.01}{\textit{#1}}}
\newcommand{\RegionMarkerTok}[1]{#1}
\newcommand{\SpecialCharTok}[1]{\textcolor[rgb]{0.00,0.00,0.00}{#1}}
\newcommand{\SpecialStringTok}[1]{\textcolor[rgb]{0.31,0.60,0.02}{#1}}
\newcommand{\StringTok}[1]{\textcolor[rgb]{0.31,0.60,0.02}{#1}}
\newcommand{\VariableTok}[1]{\textcolor[rgb]{0.00,0.00,0.00}{#1}}
\newcommand{\VerbatimStringTok}[1]{\textcolor[rgb]{0.31,0.60,0.02}{#1}}
\newcommand{\WarningTok}[1]{\textcolor[rgb]{0.56,0.35,0.01}{\textbf{\textit{#1}}}}
\usepackage{graphicx}
\makeatletter
\def\maxwidth{\ifdim\Gin@nat@width>\linewidth\linewidth\else\Gin@nat@width\fi}
\def\maxheight{\ifdim\Gin@nat@height>\textheight\textheight\else\Gin@nat@height\fi}
\makeatother
% Scale images if necessary, so that they will not overflow the page
% margins by default, and it is still possible to overwrite the defaults
% using explicit options in \includegraphics[width, height, ...]{}
\setkeys{Gin}{width=\maxwidth,height=\maxheight,keepaspectratio}
% Set default figure placement to htbp
\makeatletter
\def\fps@figure{htbp}
\makeatother
\setlength{\emergencystretch}{3em} % prevent overfull lines
\providecommand{\tightlist}{%
  \setlength{\itemsep}{0pt}\setlength{\parskip}{0pt}}
\setcounter{secnumdepth}{-\maxdimen} % remove section numbering
\ifluatex
  \usepackage{selnolig}  % disable illegal ligatures
\fi

\title{Project2}
\author{Maribeth McCook}
\date{}

\begin{document}
\maketitle

\hypertarget{part-i-conceptual-checks-active-practice}{%
\subsubsection{Part I: Conceptual Checks \& Active
Practice}\label{part-i-conceptual-checks-active-practice}}

\textbf{1.Give an example of a nominal scale measurement of some
attribute that you might want to study in your research area of
interest. Explain why it is nominal.}

Do students prefer online or in person courses? This is nominal because
there are two categories with no order, prefer online, and prefer
in-person.

\textbf{2. Give an example of an ordinal scale measurement of some
attribute that you might want to study in your research area of
interest. Explain why it is ordinal.}

Satisfaction with instruction on a ``dissatisfied,neutral, satisfied''
scale. This is ordinal because it is categories that are in a certain
order, without equal space between them.

\textbf{3. Why are data transformations sometimes used?}

If the test being used requires normally distributed data, then you can
transform non normal data into normal data and still use the test.

\textbf{4. Why do we conduct tests of significance?}

In order to see if we should retain or reject the null hypothesis for an
alternative one.

\textbf{5. What does the null hypothesis state?}

Everything is equal, there are no differences between the groups in the
study

\textbf{6. What is meant when the null hypothesis is retained?
Rejected?}

Retained - It means that the sample did not contain any significant
differences Rejected - The sample contained significant differences and
the alternative hypothesis was accepted.

\textbf{7. Is it appropriate to declare a null hypothesis such as H0: 𝜇=
b to be true if the results are ``Nonsignificant''?}

No, because there is always a chance that the sample did not include the
differences. So, the null hypothesis can never be proven.

\textbf{8. When testing a null hypothesis such as H0;𝜇= b, what can be
said about a ``significant'' result?}

One should reject the null hypothesis in favor of the alternative.

\textbf{9. Explain what an Alpha level is.} The probability that you're
rejecting the null hypothesis when it's true. Typically it is set at
.05, or a 5\% chance that you are making a mistake.

\textbf{10. What is a P-value and how does it relate to an alpha level?}

A p-value is the probability that the results you got from your sample
would happen if the null hypothesis was true. So, if the p is less than
the alpha, then the null hypothesis is considered to be false, and if p
is greater the null is retained.

\textbf{11. What does a p-value tell us, and what does it not tell us?}
It tells us the probability that we would have gotten that result if the
null hypothesis was true.It doesn't tell us the size of the effect.

\textbf{12. What is a Type 1 error? What is a Type 2 error?}

Type one is when there is a false positive result, or rejecting the null
hypothesis for the alternative when the null is true. Type 2 is when
there is a false negative, or the alternative is true but we fail to
reject the null.

\textbf{13.} Three hundred participants were randomly selected from a
large public university research participant pool of about 6,500
undergraduate students. The 300 students were randomly assigned to one
of three groups of equal size. All three groups watched a 10-minute
video and then answered a few questions. In Group 1, the video described
the health benefits of regular exercise. In Group 2, the video presented
an expert panel discussion of refugee issues. In Group 3, the video
presented an expert panel discussion of refugee issues that included
graphic images of injured children. After viewing the video, one of the
questions asked, ``Should the US increase the number of refugees it
admits into our country each year?'' The sample data are given below.

\begin{enumerate}
\def\labelenumi{\alph{enumi})}
\item
  The study population is all undergraduate students
\item
\end{enumerate}

\begin{Shaded}
\begin{Highlighting}[]
\NormalTok{m }\OtherTok{\textless{}{-}} \FunctionTok{matrix}\NormalTok{(}\FunctionTok{c}\NormalTok{(}\DecValTok{56}\NormalTok{,}\DecValTok{44}\NormalTok{,}\DecValTok{72}\NormalTok{,}\DecValTok{28}\NormalTok{,}\DecValTok{88}\NormalTok{,}\DecValTok{12}\NormalTok{),}\AttributeTok{nrow =} \DecValTok{2}\NormalTok{, }\AttributeTok{ncol =} \DecValTok{3}\NormalTok{)}
\NormalTok{table }\OtherTok{\textless{}{-}} \FunctionTok{as.table}\NormalTok{(m)}
\NormalTok{table}
\end{Highlighting}
\end{Shaded}

\begin{verbatim}
##    A  B  C
## A 56 72 88
## B 44 28 12
\end{verbatim}

\begin{Shaded}
\begin{Highlighting}[]
\FunctionTok{chisq.test}\NormalTok{(table)}
\end{Highlighting}
\end{Shaded}

\begin{verbatim}
## 
##  Pearson's Chi-squared test
## 
## data:  table
## X-squared = 25.397, df = 2, p-value = 3.056e-06
\end{verbatim}

\begin{enumerate}
\def\labelenumi{\alph{enumi})}
\setcounter{enumi}{2}
\tightlist
\item
\end{enumerate}

A Chi-square test of independence revealed a significant result between
the stimulus and student response to the survey.

\hypertarget{part-ii-your-data-your-way}{%
\subsubsection{Part II: Your Data, Your
Way}\label{part-ii-your-data-your-way}}

\begin{enumerate}
\def\labelenumi{\alph{enumi})}
\item
  This study seeks to understand if there is a relationship between
  taking pictures of slides and grades on tests in undergraduates. It is
  hypothesized that there is a relationship. Students were asked to take
  pictures of certain slides during a recorded lecture, while acting
  like it was a live course. They were then tested on the material. The
  bottom one-third of scores are put in a low achievement category, the
  middle third in mid-achievement, and the highest third in high
  achievement. The number of photos taken are also recorded, with the
  lowest third of the number of pictures taken are put in low pics, the
  middle third in mid-pics, and the highest third in high pics.
\item
  107 UCR undergraduates were asked to report the number of photos they
  took during a lecture and then were tested on the material. The data
  gathered of the total questions correct and total pictures taken are
  split into equal ordered thirds and placed in categories.Low pics,
  mid-pics, and high pics for number of pictures, and low achivment,
  mid-achievement, and high achievement for the number of questions
  answered.
\item
\end{enumerate}

\begin{Shaded}
\begin{Highlighting}[]
\NormalTok{Photo }\OtherTok{\textless{}{-}} \FunctionTok{read\_excel}\NormalTok{(}\StringTok{"Photo For R.xlsx"}\NormalTok{)}

\NormalTok{Photo}\SpecialCharTok{$}\NormalTok{TOTAL\_CORRECT }\OtherTok{\textless{}{-}} \FunctionTok{cut}\NormalTok{(Photo}\SpecialCharTok{$}\NormalTok{TOTAL\_CORRECT, }\DecValTok{3}\NormalTok{, }\AttributeTok{labels=}\FunctionTok{c}\NormalTok{(}\StringTok{\textquotesingle{}LowA\textquotesingle{}}\NormalTok{, }\StringTok{\textquotesingle{}MidA\textquotesingle{}}\NormalTok{, }\StringTok{\textquotesingle{}HighA\textquotesingle{}}\NormalTok{))}

\NormalTok{Photo}\SpecialCharTok{$}\NormalTok{photo\_total }\OtherTok{\textless{}{-}} \FunctionTok{cut}\NormalTok{(Photo}\SpecialCharTok{$}\NormalTok{photo\_total, }\DecValTok{3}\NormalTok{, }\AttributeTok{labels=}\FunctionTok{c}\NormalTok{(}\StringTok{\textquotesingle{}LowP\textquotesingle{}}\NormalTok{, }\StringTok{\textquotesingle{}MidP\textquotesingle{}}\NormalTok{, }\StringTok{\textquotesingle{}HighP\textquotesingle{}}\NormalTok{))}

\NormalTok{photo }\OtherTok{\textless{}{-}} \FunctionTok{subset}\NormalTok{(Photo,}\AttributeTok{select=}\FunctionTok{c}\NormalTok{(photo\_total,TOTAL\_CORRECT))}


\NormalTok{cp }\OtherTok{\textless{}{-}} \FunctionTok{table}\NormalTok{(photo}\SpecialCharTok{$}\NormalTok{photo\_total,photo}\SpecialCharTok{$}\NormalTok{TOTAL\_CORRECT)}
\NormalTok{cp}
\end{Highlighting}
\end{Shaded}

\begin{verbatim}
##        
##         LowA MidA HighA
##   LowP    38    5     0
##   MidP     5   38     3
##   HighP    0    3    14
\end{verbatim}

\begin{Shaded}
\begin{Highlighting}[]
\FunctionTok{chisq.test}\NormalTok{(cp)}
\end{Highlighting}
\end{Shaded}

\begin{verbatim}
## Warning in chisq.test(cp): Chi-squared approximation may be incorrect
\end{verbatim}

\begin{verbatim}
## 
##  Pearson's Chi-squared test
## 
## data:  cp
## X-squared = 126.13, df = 4, p-value < 2.2e-16
\end{verbatim}

\begin{Shaded}
\begin{Highlighting}[]
\FunctionTok{cramerV}\NormalTok{(cp)}
\end{Highlighting}
\end{Shaded}

\begin{verbatim}
## Cramer V 
##   0.7713
\end{verbatim}

The x-squared value is large, 126.13, so it is likely that the variables
are related. The p-value suggests that this variables are related, since
it is a significant result, p =\textless{} .05, Because of this, we
should reject the null hypothesis (the variables are not related) and
accept the alternative hypothesis(the variables are related). The effect
size is .7713, which is a large effect.

\begin{enumerate}
\def\labelenumi{\alph{enumi})}
\setcounter{enumi}{3}
\tightlist
\item
  A chi-squared test of independence has showed a significant result
  between number of photos taken and number of questions answered
  correctly. This means that the hypothesis of this study (that they are
  related) is supported. In order to find the direction of the
  relationship, the original numerical data should be used in a t-test,
  instead of the ordinal categories.This study can only address if there
  is a relationship.
\end{enumerate}

\hypertarget{part-iii-evaluating-the-literature}{%
\subsubsection{Part III: Evaluating the
Literature}\label{part-iii-evaluating-the-literature}}

Abstract

The present study piloted a cognitive exercise program in a college
classroom to enhance learning of lecture material. Undergraduate
students enrolled in introductory psychology (N ¼ 68) completed
variations of letter--number cancellation tasks with spoken instructions
in 5-min sessions prior to lecture during four nonconsecutive class
periods. Results showed significantly better exam performance on
material based on lectures that followed cognitive exercise compared to
lectures on non exercise days. On an anonymous program feedback survey,
students reported significantly greater levels of alertness following
cognitive exercise versus before; the majority of students rated their
attention to lecture and note-taking ability as above average after
cognitive workouts. Although preliminary, findings suggest that
cognitive exercise in the classroom may positively impact learning for
college students.

The null hypothesis is that cognitive exercise would not make a
difference on student's mind wandering during a lecture.

The null isn't stated, but I got it based on their alternative
hypothesis stating that cognitive exercise will facilitate attending a
lecture.

The results do include a p-value, but it also includes a t-test
statistic, effect size, and confidence interval as well.

They didn't really explain what the statistics meant. I think that they
assume the readers know what they are. They did explain that the t-test
showed significantly greater test scores on exercise related materials
and greater self reported attention during and after the cognitive
exercise, which was correct.

The don't specifically state that d=.624 is the effect size, but they do
correctly state that there was a moderate effect in the discussion.

Honestly, I can't think of any more analyses I would want for this
study.

\end{document}
